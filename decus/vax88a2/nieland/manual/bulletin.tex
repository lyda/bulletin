\chapter{The BULLETIN Utility}\section{Introduction} DThe Electronic Bulletin Board Utility (BULLETIN) can be used to postFbulletins on the computer for other users on your system to see.  ThisHutility also allows you to file, print, and reply to messages that otherusers have posted. HWhen a new bulletin has been posted you will be notified at login of theHnew message.  If the message is a system message, the text of the entireHmessage will be displayed.  If the message is not a system message, thenEyou will be notified of the new message and the header of the messagewill be displayed. GUsers can set parameters within BULLETIN to be notified of new messagesGimmediately, when logged on, or to have the BULLETIN system prompt themat login to read new messages. \section{Command Summary} LFollowing is a summary of BULLETIN commands, items in brackets are optional:\\[6pt]\smaller\begin{tabular} {lp{3.0in}}$\bf{}Command & \bf{}Description\rm{}\\[3pt]W\makebox[1.75in][l]{ADD [\it{}File] [/NOEDIT]\rm{}} & Add a bulletin to current folder.MSpecify either a file containing the message. The editor is used to compose aGnew message or change the specified file by default. Specify /NOEDIT toMdeactivate the editor.  If you use /NOEDIT with a specified file, the file isadded to the folder as is.\\%BACK & Back up to previous message.\\EDELETE [\it{}Msg-number] \rm{} & Delete current (last-read)message ordesignated message.OYou can only delete a message if you are the owner or have system privileges.\\4DIRECTORY   & List a summary of bulletin messages inthe current folder.\\$EXIT        & Exits from BULLETIN.\\
\end{tabular}\begin{tabular} {lp{3.0in}}I\makebox[1.75in][l]{FILE }& Copies current (last-read) message into a VMS	 file. \\0HELP & Displays information on using BULLETIN.\\;INDEX [/NEW] [/RESTART] &  Index of Folders and messages.\\8NEXT & Skips to next bulletin message and displays it.\\>PRINT & Queues the current (last-read) message for printing.\\3READ [\it{}Msg-number\rm{}] & Displays next page of6 message, the next message, or the designated message.4Pressing the \CR{} key performs the same function.\\
\end{tabular}\begin{tabular} {lp{3.0in}}J\makebox[1.75in][l]{RESPOND } & Sends a reply to the sender of the current (last read) message.\\/SEARCH \it{}Search-string\rm{} & Searches for a3 message that contains the specified text string.\\5SELECT \it{}Foldername\rm{} & To move from one folder to another within BULLETIN.\\
\end{tabular}\begin{tabular} {lp{3.0in}}K\makebox[1.75in][l]{SET NOTIFY  }& Specifies that you will be notified whenBnew messages are added to current folder when you are logged on.\\6SET READNEW & Specifies that you will be prompted upon9logging in if you wish to read new non-system messages.\\4SET LOGIN   & Specifies that you wish to be notified  of any new messages at login.\\
\end{tabular}\begin{tabular} {lp{3.0in}}J\makebox[1.75in][l]{SET ACCESS}  & Specifies access to a folder to certainMusers. Only the creator of a folder or a privileged account can set or change
the access.\\3SET FOLDER \it{}Foldername\rm{}  & To move from one folder to another.\\5SHOW \it{}Parameter\rm{} & Displays information aboutJ the parameter that was established by one of the previous set commands.\\,QUIT & Exits from BULLETIN (Same as EXIT).\\
\end{tabular}\normalsize !\section{Preset BULLETIN Folders} KThe following folders have already been created within the BULLETIN system:\smaller\\[8pt]\begin{tabular} {lp{5.5in}}%\bf{}Folder & \bf{}Description\\[3pt].APUPI\_NEWS & The Day's News From AP and UPI\\&Community & News of Community Events\\"Computers & Computer information\\"Contracts & Contract information\\=CVNET\_ARPA & Messages from CVNET (Color Vision) on ARPANet\\3GENERAL   & General purpose bulletins for system.\\1HACKERS & Computer Hackers Questions and Answer\\4HE\_ADMIN  & HE Division administration bulletins.\\3HEA\_ADMIN & HEA Branch administration Bulletins.\\3HED\_ADMIN & HED Branch administration Bulletins.\\3HEF\_ADMIN & HEF Branch administration Bulletins.\\3HEG\_ADMIN & HEG Branch administration Bulletins.\\3HET\_ADMIN & HET Branch administration Bulletins.\\,HEX & HEX Branch administration Bulletins.\\;HEX-CAT\_ADMIN & HEX-CAT Branch administration Bulletins.\\,Military & Messages for Military Personnel\\FN-Language\_ARPA & Messages from NL-KR (Natural Language) on ARPANet\\0Neuron\_ARPA & Messages from NEURON on ARPANet\\4Psychnet\_ARPA & Messages from PSYCHNET on ARPANet\\:RCB & Resource Control Board and Secretaries Information\\*Secretaries & Secretaries Bulletin Board\\%System & \Computer System Bulletins\\Training & Training news\\0Vision\_ARPA & Messages from VISION on ARPANet\\
\end{tabular}\normalsize BUsers may post bulletins in the GENERAL folder.  Branch Chiefs andDsecretaries may also post messages in their respective ADMIN folder.KThe Division chief and secretary may post messages in the HE\_ADMIN folder. HUsers may read any bulletin in GENERAL, HACKERS, SYSTEM, and APUPI\_NEWSIfolders and any of the \_\_ARPA folders. HE Division members may read theFbulletins within HE\_ADMIN and their respective branch's ADMIN folder. !\section{Examples Using BULLETIN} LThe following examples illustrate the use of the BULLETIN utility for common
applications.\begin{enumerate}J\item User invokes the BULLETIN utility and reads the two messages.\\[2pt] \smaller\tt
\begin{quote}\$ {\it{}BULLETIN }\\)Type READ to read new general messages.\\BULLETIN\tt{}>{\it{} READ }\\Message number:\mbox{   }1\\7Description: INFO ON HOW TO USE THE BULLETIN UTILITY.\\DFrom: SYSTEM\mbox{       }Date:  6-MAR-1987 Permanent message\\[4pt]9This message is being displayed by the BULLETIN facility.BThis is a non-DEC facility, so it is not described in the manuals.@Messages can be submitted by using the BULLETIN command.  SystemBmessages, such as this one, are displayed in full, but can only beCentered by privileged users.  Non-system messages can be entered byBanyone, but only their topics will be displayed at login time, andBwill be prompted to optionally read them.  (This prompting featureAcan be disabled).  All bulletins can be reread at any time unlessAthey are deleted or expire. For more information, see the on-linehelp (via HELP BULLETIN).\\ BULLETIN\tt{}> {\it{}READ} \\Message number:\mbox{   }2\\CDescription: INFO ON BEING PROMPTED TO READ NON-SYSTEM BULLETINS.\\?From: SYSTEM \mbox{      }Date:  6-MAR-1987 Permanent message\\:Non-system bulletins (such as this) can be submitted by anHuser. Users are alerted at login time that new non-system bulletins have-been added, but only their topics are listed.FOptionally, users can be prompted at login time to see if they wish toCread the bulletins.  When reading the bulletins in this manner, the>bulletins can optionally be written to a file. If you have theAsubdirectory [.BULL] created, BULLETIN will use that directory as/the default directory to write the file into.\\ \normalsizeK\rm A user can disable the READNEW prompting featuring by using BULLETIN asfollows: \\[1em]\smaller\$ {\it{} BULLETIN }\\,\tt BULLETIN \tt{}> {\it{} SET NOREADNEW }\\$BULLETIN \tt{}> {\it{} EXIT }\\[3pt]C\rm Afterwards, the user will only be alerted of the bulletins, andGwill have to use the BULLETIN utility in order to read the messages. \\#\tt BULLETIN \tt{}> {\it{} EXIT }\\\end{quote}\normalsize'\item Login Prompt using SET READNEW:\\ \nopagebreak{}\smaller\tt\begin{tabular} { l l l c }\multicolumn{4}{l}L{************************New GENERAL messages***********************}\\[2pt]=\underline{Description} & \underline{From} & \underline{Date}& \underline{Number}\\?Testing of bulletin system SET READNEW  & SYSTEM & 10-MAR & 3\\\multicolumn{4}{l}L{*******************************************************************}\\[2pt]\multicolumn{4}{l}G{Read messages? Type N, Q, message number, or any other key for yes:}\\\end{tabular}\\[2pt]\end{enumerate}\normalsize\rm 7\subsection{Defering notification of BULLETIN Messages} KYou can defer the notification of BULLETIN message that is normally done atFlogin time by placing the following line in your {\tt LOGIN.COM} file: 2\begin{center} \$ CHECK\_BULL :== OFF \end{center} BThis turns ``off'' the BULLETIN notification system.  You can then8check for bulletins at any time by entering the command: -\begin{center} \$ BULLETIN/LOGIN \end{center} 9and the system will prompt you to read any new bulletins.GYou should make sure to check for bulletins regularly or some importantmessages could be missed. \subsection{Adding Messages} MTo add a message to any bulletin folder, first select the folder to which theKmessage is to be added.  Then enter the command {\bf ADD}.  The system willMthen ask for an expiration date and a description of the message.   After theOdescription has been added, the system will activate the editor (unless /NOEDITHwas specified) and you can compose the message.  If a file was specifiedKwith the {\bf ADD} command, then the specified file will be pulled into theNeditor to be edited (unless /NOEDIT was specified, in which case the file will-be posted immediately to the bulletin board). MThe following is an example of posting a message to the GENERAL folder of theObulletin board with an expiration date of 10 days from the current time.\\[1em]	\noindent
{\smaller \tt\$ {\it BULLETIN}\CR \\[.5em]!Folder has been set to GENERAL.\\#BULLETIN$>$ {\it Add/NOEDIT} \CR \\HIt is  7-MAR-1988 16:21:44.20. Specify when the message should expire:\\JEnter absolute time: [dd-mmm-yyyy] hh:mm:ss or delta time: dddd hh:mm:ss\\{\it 10}\CR \\;Enter description header.  Limit header to 53 characters.\\{\it Test message}\CR \\4Enter message: End with ctrl-z, cancel with ctrl-c\\#{\it This is a test message.}\CR \\{\bf CTRL-Z }\\BULLETIN$>$ {\it exit}\CR \\}