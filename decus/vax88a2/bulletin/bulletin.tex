\chapter{The BULLETIN Utility}
\section{Introduction}
 
The Electronic Bulletin Board Utility (BULLETIN) can be used to post
bulletins on the computer for other users on your system to see.  This
utility also allows you to file, print, and reply to messages that other
users have posted.
 
When a new bulletin has been posted you will be notified at login of the
new message.  If the message is a system message, the text of the entire
message will be displayed.  If the message is not a system message, then
you will be notified of the new message and the header of the message
will be displayed.
 
Users can set parameters within BULLETIN to be notified of new messages
immediately, when logged on, or to have the BULLETIN system prompt them
at login to read new messages.
 
\section{Command Summary}
 
Following is a summary of BULLETIN commands, items in brackets are optional:
\\[6pt]
\smaller
\begin{tabular} {lp{3.0in}}
\bf{}Command & \bf{}Description\rm{}
\\[3pt]
\makebox[1.75in][l]{ADD [\it{}File] [/NOEDIT]\rm{}} & Add a bulletin to current folder.
Specify either a file containing the message. The editor is used to compose a
new message or change the specified file by default. Specify /NOEDIT to
deactivate the editor.  If you use /NOEDIT with a specified file, the file is
added to the folder as is.\\
BACK & Back up to previous message.\\
DELETE [\it{}Msg-number] \rm{} & Delete current (last-read)message or
designated message.
You can only delete a message if you are the owner or have system privileges.\\
DIRECTORY   & List a summary of bulletin messages in
the current folder.\\
EXIT        & Exits from BULLETIN.\\
\end{tabular}
\begin{tabular} {lp{3.0in}}
\makebox[1.75in][l]{FILE }& Copies current (last-read) message into a VMS
 file. \\
HELP & Displays information on using BULLETIN.\\
INDEX [/NEW] [/RESTART] &  Index of Folders and messages.\\
NEXT & Skips to next bulletin message and displays it.\\
PRINT & Queues the current (last-read) message for printing.\\
READ [\it{}Msg-number\rm{}] & Displays next page of
 message, the next message, or the designated message.
Pressing the \CR{} key performs the same function.\\
\end{tabular}
\begin{tabular} {lp{3.0in}}
\makebox[1.75in][l]{RESPOND } & Sends a reply to the sender of the current
 (last read) message.\\
SEARCH \it{}Search-string\rm{} & Searches for a
 message that contains the specified text string.\\
SELECT \it{}Foldername\rm{} & To move from one folder
 to another within BULLETIN.\\
\end{tabular}
\begin{tabular} {lp{3.0in}}
\makebox[1.75in][l]{SET NOTIFY  }& Specifies that you will be notified when
new messages are added to current folder when you are logged on.\\
SET READNEW & Specifies that you will be prompted upon
logging in if you wish to read new non-system messages.\\
SET LOGIN   & Specifies that you wish to be notified
 of any new messages at login.\\
\end{tabular}
\begin{tabular} {lp{3.0in}}
\makebox[1.75in][l]{SET ACCESS}  & Specifies access to a folder to certain
users. Only the creator of a folder or a privileged account can set or change
the access.\\
SET FOLDER \it{}Foldername\rm{}  & To move from one
 folder to another.\\
SHOW \it{}Parameter\rm{} & Displays information about
 the parameter that was established by one of the previous set commands.\\
QUIT & Exits from BULLETIN (Same as EXIT).\\
\end{tabular}
\normalsize
 
\section{Preset BULLETIN Folders}
 
The following folders have already been created within the BULLETIN system:
\smaller
\\[8pt]
\begin{tabular} {lp{5.5in}}
\bf{}Folder & \bf{}Description\\[3pt]
APUPI\_NEWS & The Day's News From AP and UPI\\
Community & News of Community Events\\
Computers & Computer information\\
Contracts & Contract information\\
CVNET\_ARPA & Messages from CVNET (Color Vision) on ARPANet\\
GENERAL   & General purpose bulletins for system.\\
HACKERS & Computer Hackers Questions and Answer\\
HE\_ADMIN  & HE Division administration bulletins.\\
HEA\_ADMIN & HEA Branch administration Bulletins.\\
HED\_ADMIN & HED Branch administration Bulletins.\\
HEF\_ADMIN & HEF Branch administration Bulletins.\\
HEG\_ADMIN & HEG Branch administration Bulletins.\\
HET\_ADMIN & HET Branch administration Bulletins.\\
HEX & HEX Branch administration Bulletins.\\
HEX-CAT\_ADMIN & HEX-CAT Branch administration Bulletins.\\
Military & Messages for Military Personnel\\
N-Language\_ARPA & Messages from NL-KR (Natural Language) on ARPANet\\
Neuron\_ARPA & Messages from NEURON on ARPANet\\
Psychnet\_ARPA & Messages from PSYCHNET on ARPANet\\
RCB & Resource Control Board and Secretaries Information\\
Secretaries & Secretaries Bulletin Board\\
System & \Computer System Bulletins\\
Training & Training news\\
Vision\_ARPA & Messages from VISION on ARPANet\\
\end{tabular}
\normalsize
 
Users may post bulletins in the GENERAL folder.  Branch Chiefs and
secretaries may also post messages in their respective ADMIN folder.
The Division chief and secretary may post messages in the HE\_ADMIN folder.
 
Users may read any bulletin in GENERAL, HACKERS, SYSTEM, and APUPI\_NEWS
folders and any of the \_\_ARPA folders. HE Division members may read the
bulletins within HE\_ADMIN and their respective branch's ADMIN folder.
 
\section{Examples Using BULLETIN}
 
The following examples illustrate the use of the BULLETIN utility for common
applications.
\begin{enumerate}
\item User invokes the BULLETIN utility and reads the two messages.\\[2pt]
 
\smaller
\tt
\begin{quote}
\$ {\it{}BULLETIN }\\
Type READ to read new general messages.\\
BULLETIN\tt{}>{\it{} READ }\\
Message number:\mbox{   }1\\
Description: INFO ON HOW TO USE THE BULLETIN UTILITY.\\
From: SYSTEM\mbox{       }Date:  6-MAR-1987 Permanent message\\[4pt]
This message is being displayed by the BULLETIN facility.
This is a non-DEC facility, so it is not described in the manuals.
Messages can be submitted by using the BULLETIN command.  System
messages, such as this one, are displayed in full, but can only be
entered by privileged users.  Non-system messages can be entered by
anyone, but only their topics will be displayed at login time, and
will be prompted to optionally read them.  (This prompting feature
can be disabled).  All bulletins can be reread at any time unless
they are deleted or expire. For more information, see the on-line
help (via HELP BULLETIN).\\
 
BULLETIN\tt{}> {\it{}READ} \\
Message number:\mbox{   }2\\
Description: INFO ON BEING PROMPTED TO READ NON-SYSTEM BULLETINS.\\
From: SYSTEM \mbox{      }Date:  6-MAR-1987 Permanent message\\
Non-system bulletins (such as this) can be submitted by an
user. Users are alerted at login time that new non-system bulletins have
been added, but only their topics are listed.
Optionally, users can be prompted at login time to see if they wish to
read the bulletins.  When reading the bulletins in this manner, the
bulletins can optionally be written to a file. If you have the
subdirectory [.BULL] created, BULLETIN will use that directory as
the default directory to write the file into.\\
 
\normalsize
\rm A user can disable the READNEW prompting featuring by using BULLETIN as
follows: \\[1em]
\smaller
\$ {\it{} BULLETIN }\\
\tt BULLETIN \tt{}> {\it{} SET NOREADNEW }\\
BULLETIN \tt{}> {\it{} EXIT }\\[3pt]
\rm Afterwards, the user will only be alerted of the bulletins, and
will have to use the BULLETIN utility in order to read the messages. \\
\tt BULLETIN \tt{}> {\it{} EXIT }\\
\end{quote}
\normalsize
\item Login Prompt using SET READNEW:\\
 
\nopagebreak{}
\smaller\tt
\begin{tabular} { l l l c }
\multicolumn{4}{l}
{************************New GENERAL messages***********************}\\[2pt]
\underline{Description} & \underline{From} & \underline{Date}
& \underline{Number}\\
Testing of bulletin system SET READNEW  & SYSTEM & 10-MAR & 3\\
\multicolumn{4}{l}
{*******************************************************************}\\[2pt]
\multicolumn{4}{l}
{Read messages? Type N, Q, message number, or any other key for yes:}\\
\end{tabular}\\[2pt]
\end{enumerate}
\normalsize\rm
 
\subsection{Defering notification of BULLETIN Messages}
 
You can defer the notification of BULLETIN message that is normally done at
login time by placing the following line in your {\tt LOGIN.COM} file:
 
\begin{center} \$ CHECK\_BULL :== OFF \end{center}
 
This turns ``off'' the BULLETIN notification system.  You can then
check for bulletins at any time by entering the command:
 
\begin{center} \$ BULLETIN/LOGIN \end{center}
 
and the system will prompt you to read any new bulletins.
You should make sure to check for bulletins regularly or some important
messages could be missed.
 
\subsection{Adding Messages}
 
To add a message to any bulletin folder, first select the folder to which the
message is to be added.  Then enter the command {\bf ADD}.  The system will
then ask for an expiration date and a description of the message.   After the
description has been added, the system will activate the editor (unless /NOEDIT
was specified) and you can compose the message.  If a file was specified
with the {\bf ADD} command, then the specified file will be pulled into the
editor to be edited (unless /NOEDIT was specified, in which case the file will
be posted immediately to the bulletin board).
 
The following is an example of posting a message to the GENERAL folder of the
bulletin board with an expiration date of 10 days from the current time.\\[1em]
\noindent
{\smaller \tt
\$ {\it BULLETIN}\CR \\[.5em]
Folder has been set to GENERAL.\\
BULLETIN$>$ {\it Add/NOEDIT} \CR \\
It is  7-MAR-1988 16:21:44.20. Specify when the message should expire:\\
Enter absolute time: [dd-mmm-yyyy] hh:mm:ss or delta time: dddd hh:mm:ss\\
{\it 10}\CR \\
Enter description header.  Limit header to 53 characters.\\
{\it Test message}\CR \\
Enter message: End with ctrl-z, cancel with ctrl-c\\
{\it This is a test message.}\CR \\
{\bf CTRL-Z }\\
BULLETIN$>$ {\it exit}\CR \\
}
